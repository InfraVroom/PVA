\section{Project-activiteiten}
Gedurende de loop van het project zijn enkele testmomenten waarvoor doelstellingen zijn gesteld waar naartoe zal moeten worden gewerkt. Voor elk testmoment zal een presentatie gemaakt, voorbereidt en gehouden moeten worden. Ook moet er elk testmoment een demonstratie worden gegeven. 

\subsection{Taken voor testmoment 1}
De eerste doelstelling is: “De Smart Car is in staat om een statisch en bewegend doel te vinden en ernaar toe te rijden.”. Om doelstelling 1 te halen zullen de sensoren het doel dus moeten kunnen detecteren en gegevens doorgeven. De code zal ervoor moeten zorgen dat de motoren het autootje naar het doel laten rijden. Hiervoor zal rijden en sturen mogelijk moeten zijn. Om gericht naar het object te kunnen rijden, moet het autootje recht kunnen rijden. Hiervoor moeten eerst de motoren gekalibreerd worden. Zodra de motoren allemaal even snel draaien, kan er een code geschreven worden die het autootje aanstuurt. 
Na testmoment 1 zal ook de eerste tussendocumentatie gemaakt en ingeleverd moeten worden. 

\subsection{Taken voor testmoment 2}
De tweede doelstelling is: “Het robotwagentje is in staat om een statisch en bewegend doel te vinden terwijl er obstakels aanwezig zijn en het robotwagentje kan naar het doel toe te rijden terwijl het de obstakels vermijdt.”. Om dit te realiseren zullen de IR- en SONAR-sensoren gebruikt moeten worden om de obstakels te detecteren. De waardes die worden gemeten moeten vervolgens gebruikt worden in een code die het autootje om de obstakels stuurt. Hierbij moet gedacht worden aan de verschillende manieren van zijwaarts verplaatsen (draaien en rechtdoor of zijwaartse functie van de wielen).  

\subsection{Taken voor testmoment 3}
In de week van testmoment 3 zal het eindverslag geschreven en ingeleverd moeten worden. 
De derde doelstelling is: “Het robotwagentje voldoet aan doelstellingen 1 en 2 en is in staat om op basis van een bewegend doel ontwijkend gedrag te vertonen. Dus in dit scenario is het robotwagentje de “prooi” en het bewegend doel eigenlijk de “jager”.” Om dit doel te bereiken zal het autootje dus instaat moeten zijn om de “jager” te detecteren en hiervan weg te rijden zonder tegen obstakels te rijden. Hierbij moet ook gedacht worden aan hoe het kan worden voorkomen dat het autootje vast komt te zitten. 

\subsection{Taken voor testmoment 4}
De vierde doelstelling is: “Het robotwagentje in “jager-modus” is in staat om een ander robotwagentje op te sporen en te vangen (aan te tikken). Het robotwagentje in “prooi-modus” is in staat om (zo lang mogelijk) uit handen te blijven van een ander robotwagentje. Om dit te realiseren hoeft de “prooi-modus” die voor testmoment 3 al gemaakt is alleen maar verbeterd te worden. Om de “jager-modus” te maken moet het in plaats van weg rijden van het andere autootje, er juist naartoe rijden. Hierbij is het belangrijk te denken aan snelheid en eventueel ook efficiëntie van bijvoorbeeld de route die het autootje neemt. Ook zal het nodig zijn makkelijk tussen deze twee modi te kunnen wisselen. 
\\\\
Tijdens het proces van het realiseren hiervan zullen meerdere meetings plaatsvinden, zowel met als zonder de docentbegeleider. Bij deze meetings zullen de gang van zaken besproken worden en kunnen vragen gesteld worden. 
