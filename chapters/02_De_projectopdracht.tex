\section{De projectopdracht}
\subsection{Project naam}
De project-groepsnaam is \textit{Infra Vroom}. Dit is gebaseerd op een \textbf{IR sensor} en een godsdienstige betekenis.

De \textit{infrarood sensor} is de sensor die het meest wordt gebruikt op de zelf rijdende auto. Met behulp van deze sensor kan bepaald worden of een object binnen de bepaalde marge komt. Zodra dit het geval is, zal de IR sensor een positief signaal afgeven.

De naam vroom is afgeleid van het geluid dat een brandstof auto maakt. Ook betekend vroom gelovig, het team Infra Vroom gelooft er heilig in dat de zelf rijdende auto de eerste plaats zal behalen.

\subsection{Probleem}
Het probleem van de opdrachtgever is dat hij een zelf rijdende auto nodig heeft. Deze auto moet in staat zijn objecten te volgen en te ontwijken.

\subsection{Doelstelling}
De doelstelling is het creëren van een zelf rijdende auto om zo het probleem van de opdrachtgever op te lossen. Ook is het doel, leren hoe algoritmes kunnen worden ontwerpen en toegepast, en efficiënter worden gecodeerd.

\subsection{Resultaat}
De zelf rijdende auto zal in staat zijn om objecten te ontwijken en te volgen. Dit zal het doen door de sensor data te verwerken in een algoritme. Dit algoritme zal in staat zijn om keuzes te maken. Deze keuzes zullen leiden tot een bepaalde verplaatsing van de auto.

\subsection{Eisen}
Er moet aan de volgende eisen worden voldaan:
\begin{itemize}
   \item Object detecteren en verwerken
   \item Object volgen en/of ontwijken
   \item Snel reageren in situatie > beter volgen of ontwijken van objecten
 \end{itemize}

\subsection{Probleem oplossing}
De auto zal in staat zijn om op zichzelf te rijden, dit zal het doen door statische maar ook dynamische objecten te ontwijken. 
Ook zal de auto in staat zijn om van A naar B te reizen, terwijl B een dynamische locatie/object is.
Om verkeersveiligheid te bevorderen zal onderlinge communicatie tussen de auto's vereist zijn. Zo kunnen alle auto's samenwerken.