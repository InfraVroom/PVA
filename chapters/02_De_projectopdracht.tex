\section{De projectopdracht}
\subsection{Project naam}
Onze projectnaam is \textit{Infra Vroom}. Dit is gebaseerd op een \textbf{IR sensor} en een godsdienstige betekenis.

De \textit{infrarood sensor} is de sensor die wij het meest gebruiken op onze zelf rijdende auto. Met behulp van deze sensor kunnen wij bepalen of  een object binnen de door ons bepaalde marge komt. Zodra dit het geval is, zal de IR sensor een positief signaal afgeven.

De naam vroom hebben wij afgeleid van het geluid dat een brandstof auto maakt. Ook betekend vroom gelovig, als team geloven wij er heilig in dat onze zelf rijdende auto de eerste plaats zal behalen.

\subsection{Probleem}
Het probleem van onze opdrachtgever is dat hij een zelf rijdende auto nodig heeft. Deze auto moet in staat zijn objecten te volgen en te ontwijken.

\subsection{Doelstelling}
Onze doelstelling is het creëren van een zelf rijdende auto om zo het probleem van onze opdrachtgever op te lossen. Ook zullen wij leren hoe wij algoritmes kunnen ontwerpen en toepassen, en efficiënter coderen.

\subsection{Resultaat}
Onze zelf rijdende auto zal in staat zijn om objecten te ontwijken en te volgen. Dit zal hij doen door de sensor data te verwerken in een algoritme. Dit algoritme zal in staat zijn om keuzes te maken. Deze keuzes zullen leiden tot een bepaalde verplaatsing van de auto.

\subsection{Eisen}
Wij houden ons aan de volgende eisen:

\textit{- Object detecteren en verwerken}

\textit{- Object volgen en/of ontwijken}

\textit{- Snel reageren in situatie > beter volgen of ontwijken van objecten}

\subsection{Probleem oplossing}
Onze auto zal in staat zijn om op zichzelf te rijden, dit zal hij doen door\\ statische maar ook dynamische objecten te ontwijken. 
Ook zal de auto in staat zijn om van A naar B te reizen, terwijl B een dynamische locatie/object is.
Om verkeersveiligheid te bevorderen zal onderlinge communicatie tussen de auto's vereist zijn. Zo kunnen alle auto's samenwerken.