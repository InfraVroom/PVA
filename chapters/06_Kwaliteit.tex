\section{Kwaliteit}
Tijdens het maken van de (tussen)producten zal er telkens gekeken worden naar de eisen waaraan het moet voldoen en hier naartoe worden gewerkt. Mocht iets niet lukken, dan zal er hulp worden ingeschakeld van de technisch adviseur (Ad van den Bergh).

Gedurende het realisatie proces zullen er ook regelmatig vragen worden gesteld aan de opdrachtgever, om er zo zeker van te zijn dat het product naar wens wordt ontworpen en gemaakt.

Een andere manier waarop de kwaliteit wordt gewaarborgd is door het gebruik van goede programma’s. De programma’s die gebruikt worden zijn met name, Arduino IDE en KiCad. Alle code zal in de Arduino IDE omgeving geschreven worden. Dit zorgt ervoor dat de code direct op het Arduino bordje geüpload en getest kan worden.
